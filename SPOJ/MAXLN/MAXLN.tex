\documentclass[a4paper,10pt]{article}
\usepackage[utf8x]{inputenc}
\usepackage{amssymb,amsmath}

%opening
\title{Explanation of SPOJ Problem MAXLN}
\author{Tapasweni Pathak}

\begin{document}

\begin{center}
\Large\textbf{Explanation of SPOJ Problem MAXLN}\\
\end{center}

\raggedright

First of all the triangle CAB shown in the figure is right angled triangle.So, now we have :

\begin{eqnarray}
AB^2 + AC^2 = BC^2 \\
AB^2 + AC^2 = (2r)^2 \\
AB^2 = 4r^2 - AC^2 
\end{eqnarray}

As given in the question :

\begin{eqnarray}
s = AB^2 + AC\\
s = 4r^2 - AC^2 +AC\\
s = -AC^2 + AC + 4r^2
\end{eqnarray}

Now we have a quadratic equation, and we have to find its maximum. Maximum value occurs when

\begin{flalign}
AC = \frac{-b}{2a}
\end{flalign}

So now using values of a and b from equation 6 we get :

\begin{flalign}
AC &= \frac{-b}{2a} \\
\nonumber &= \frac{-1}{-2}\\
\nonumber &= \frac{1}{2}
\end{flalign}

Putting the value of AC from 8 in 6 we get :

\begin{flalign}
s &= -\frac{1}{2}^2 + \frac{1}{2} + 4r^2\\
\nonumber &= \frac{-1}{4} + \frac{1}{2} + 4r^2\\
\nonumber &= \frac{1}{4} + 4r^2\\
\nonumber &= 0.25 + 4r^2\\
\end{flalign}

Hence the equation we are going to use is :

\begin{equation}
s = 0.25 + 4r^2
\end{equation}

\end{document}
